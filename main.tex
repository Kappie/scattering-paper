\documentclass[twocolumn, openany, oneside, article]{memoir}

\usepackage[style=numeric-comp, backend=biber]{biblatex}
\usepackage{bm}
\usepackage{amsmath}
\usepackage{amsfonts}
\usepackage{csquotes}
\usepackage{hyperref}

\bibliography{references}

\title{Mijn titel}
\author{Geert Kapteijns}
\date{\today}

\begin{document}
\maketitle
\pagestyle{simple}

\begin{abstract}

Lorem ipsum dolor sit amet, consectetur adipisicing elit, sed do eiusmod
tempor incididunt ut labore et dolore magna aliqua. Ut enim ad minim veniam,
quis nostrud exercitation ullamco laboris nisi ut aliquip ex ea commodo
consequat. Duis aute irure dolor in reprehenderit in voluptate velit esse
cillum dolore eu fugiat nulla pariatur. Excepteur sint occaecat cupidatat non
proident, sunt in culpa qui officia deserunt mollit anim id est laborum.

\end{abstract}

\chapter{Introduction}

Let me start off by saying this is an informal document meant to make my
research efforts of the last three months accessible. I have made no huge effort
to cite the first paper to establish a concept or even to cite well-established
results at all.

The challenge at hand is, in the broadest sense, to use statistical models and computational power to improve outcome
for ischemic stroke patients. This note focuses on using image data to aid radiologists in the process of deciding, in
the earliest possible stage, whether to use intra-arterial therapy (a relatively new endovascular, catheter-based
treatment.)

I have limited my study to finding image features in 3D NCCT scans that are relevant for this decision. I believe that,
despite popularity and exceptional engineering results in many areas (easily outperforming \enquote{hand-crafted}
algorithms), deep learning is not the best tool for this problem (and I suspect many other problems in the medical
domain), because medical images are high-dimensional, training data is scarce and there is not always a clearly defined
ground truth.

What I have tried here is to use a different approach, namely one where no model is trained for feature extraction, but
which uses an image representation called the scattering transform, that is known to have certain desirable properties,
e.g. it preserves high-frequency information, is translation invariant over a tunable window and stable to deformations.
This image representation may, in association with a simple classifier like an SVM, be used in classification problems.

In the classification experiments, I have chosen to focus on classifying brain hemispheres as being unaffected or
affected by stroke, using the MRCLEAN registry. While this is not a clinically relevant problem (classifying which
hemispheres is affected is a trivial task in most cases?) it serves as a test for the scattering representation, as at
least there is an unequivocal ground truth. Though not done in this work, the learned features may be visualised to aid
radiologists in giving an ASPECTS score or otherwise diagnosing the seriousness of the stroke. It is also possible, if a
good delineation of the ASPECTS regions is available, to directly train a classifier on the scattering representation of
these regions (though expert's opinion on whether a region is affected or not is not unequivocal, making it difficult to
attain super-expert performance).


\chapter{Scattering transform}
I will in the briefest possible way state what the scattering transform is. For details
refer to \cite{anden2014deep, bruna2013invariant}.

A wavelet transform is defined by convolving a signal $d$-dimensional $y(\bm{x})$ with scaled and rotated versions of a
mother wavelet $\psi_{a^j, r}(\bm{x})$, with $j \in \mathbb{Z}$ and rotations $r \in SO(d)$ (rotation group in $d$
dimensions). $d = 3$ in our case, since we are dealing with 3D images. $a = 2$ is common for image analysis. We will
describe in the next section which mother wavelet we use in practice, and how to correctly choose a finite number of
length scales and rotations $r$ (which is not trivial in 3D).

A wavelet of dilation $a^j$ and orientation $r$ looks like
\begin{equation}
  \psi_{a^j, r}(\bm{x}) = a^{-dj} \psi(a^{-j}r\bm{x})
\end{equation}
where the normalisation $a^{-dj}$ is chosen such that the energy of the mother wavelet is
conserved
\begin{equation}
  \int_{\mathbb{R}^d} d\bm{x} | \psi_{a^j, r}(\bm{x}) | = \int_{\mathbb{R}^d} d\bm{x} | \psi(\bm{x}) |.
\end{equation}
Translationally invariant coefficients (called scattering coefficients) of $y(\bm{x})$ that are stable to small deformations are obtained by taking the modulus and taking a spatial average:
\begin{equation}
    \left\| y \star \psi_{a^j, r} \right\|_{1} = \int d\bm{x} |y \star \psi_{a^j, r}|.
\end{equation}
The signals $|y \star \psi_{a^j, r}|$ are themselves unstable, and in averaging (or equivalently, removing all non-zero
frequencies) of $|y \star \psi_{a^j, r}|$ information is lost. To remedy this, we can perform a second wavelet
transform on all first-order transforms, yielding second-order scattering coefficients
\begin{equation}
  \left\| |y \star \psi_{a^{j_1}, r_1}| \star \psi_{a^{j_2}, r_2} \right\| = \int d\bm{x} | |y \star \psi_{a^{j_1}, r_1}| \star \psi_{a^{j_2}, r_2} |
\end{equation}
for all $j_1, j_2, r_1, r_2$. One can keep iterating this transform to recover more lost information in the form of
higher-order coefficients, but in practice two layers is sufficient for most tasks (luckily so, because the
computational resources required scales exponentially in the number of layers).

To make notation easier, we define $\lambda = (j, r)$ and
\begin{equation}
  U[\lambda_1, \dots \lambda_m]y = \left| \left| \left| y \star \psi_{\lambda_1} \right| \star \psi_{\lambda_2} \right|
  \dots \psi_{\lambda_m} \right|
\end{equation}
i.e. an unaveraged signal at the $m$th layer. The scattering coefficients
at the $m$th layer are then written
\begin{equation}
  \left\{ \bar{S}y(\lambda_1, \dots \lambda_m) = \int d\bm{x} U[p]y \right\}_{\lambda_i \in \Lambda}
\end{equation}

In practice it is often better not to average over the entire signals $U[p]$ (where we write $p = \lambda_1, \dots
\lambda_m$ a \emph{path}), but compute coefficients that are approximately translation invariant over lengths $a^J$.
This is achieved by computing the wavelet transforms only at scales $j \leq J$ and averaging with a low-pass (blurring) filter (in
practice a Gaussian) of support $a^J$, denoted by $\phi_{a^J}$
\begin{equation}
  S[p]y(\bm{x}) = (U[p] \star \phi_{a^J})(\bm{x}).
\end{equation}
$\left\{ S[\lambda_1, \dots, \lambda_m] \right\}_{\lambda_i \in \Lambda}$ are called the windowed scattering
coefficients at layer $m$. The correct maximum length scale $a^J$ should be chosen based on knowledge about the input
data or by cross-validation.

\chapter{Implementation details}
What follows are some considerations for choosing the mother wavelet and
rotations $r$.

\section{Construction of the mother wavelet}

For the scattering representation to be stable to additive noise and contain all
high frequency information, it must satisfy the Littlewood-Paley condition
\begin{equation}\label{eq:lp_condition}
  (1 - \epsilon) \leq A(\bm{\omega}) \leq 1 \qquad \forall \bm{\omega} \in \mathbb{R}^d
\end{equation}
with
\begin{equation}
  A(\bm{\omega}) = \left| \hat{\phi}_{a^J}(\bm{\omega}) \right|^2 + \frac{1}{2} \sum_{j \leq J} \sum_{r \in R}
  \left( \left| \hat{\psi}_{a^j, r}(\bm{\omega}) \right|^2 + \left| \hat{\psi}_{a^j, r}(\bm{-\omega}) \right|^2 \right)
\end{equation}
and $\epsilon$ small. Since the low-pass filter $\phi_{a^J}$ is normalized ($\int d\bm{x} \phi_{a^J}(\bm{x}) = 0$ or
equivalently $\hat{\phi}_{a^J}(\omega = 0) = 1$), the above condition implies
\begin{equation}\label{eq:averages_to_zero} \hat{\psi}_{a^j, r}(0) = 0 \end{equation} or equivalently: all wavelets
should average to zero.
Furthermore, if we define a mother wavelet, called the Morlet wavelet, as follows
\begin{equation}
  \psi(\bm{x}) = \mathcal{N} g_{\sigma}(\bm{x})\left( e^{i \xi \bm{x}} - \kappa_{\sigma} \right)
\end{equation}
where $g(\bm{x})_{\sigma}$ is a Gaussian and $\kappa_{\sigma} \ll 1$ has to be chosen to satisfy
\autoref{eq:averages_to_zero}, it will have the property that its Fourier transform is real, hence that if the input
signal $y$ is real, $U[j, r]y = U[j, -r]y$, allowing us to only consider positive rotations.

Instead of labeling a wavelet by its standard deviation in the spatial domain $\sigma$, it is in this case
more insightful to label it by its bandwidth $b$ in the Fourier domain, defined by
\begin{equation}
  \hat{g}_{\sigma}\left(\pm \left(\frac{b}{2}, 0, 0 \right) \right) = \exp \left( -\frac{1}{2}\sigma^2 \left(\frac{b}{2}\right)^2 \right) = \frac{1}{\sqrt{2}}
\end{equation}
leading to
\begin{equation}
  b^2 = \frac{4 \ln 2}{\sigma^2}.
\end{equation}

The Fourier transform of the Morlet is
\begin{equation}
  \hat{\psi}(\bm{\omega}) = \mathcal{N} \left( \hat{g}_{\sigma}(\bm{\omega} - \bm{\xi}) -
  \kappa_{\sigma}\hat{g}_{\sigma}(\bm{\omega}) \right) .
\end{equation}
It is, apart from the small factor $\kappa_{\sigma}$, centered at
$\bm{\xi} = (\xi, 0, 0)$ with bandwidth $b$.
The requirement $\hat{\psi}(\bm{0}) = 0$ leads to
\begin{equation}
  \kappa_{\sigma} = \frac{\hat{g}_{\sigma}(-\bm{\xi})}{\hat{g}_{\sigma}(\bm{0})}.
\end{equation}
The dilated and scaled wavelet becomes in the Fourier domain
\begin{equation}
  \hat{\psi}_{a^j, r}(\bm{\omega}) = a^{-dj} \mathcal{N} \left( \hat{g}_{\sigma}(a^j r^{-1} \bm{\omega} - \bm{\xi}) - \kappa_{\sigma}\hat{g}_{\sigma}(a^j r^{-1} \bm{\omega}) \right)
\end{equation}
so that it is (apart from the small corrective term $\kappa_{\sigma}$) centered at frequency $\bm{\omega_c} =
a^{-j}r\bm{\xi}$ with bandwidth $b_{a^j} = a^{-j}b$. Note that the corrective factor $\kappa_{\sigma}$ is invariant
under dilation and rotation.

We choose the normalisation factor of the mother wavelet $\mathcal{N}$,
in order to be able satisfy \autoref{eq:lp_condition}, as the maximum of the \enquote{raw} Littlewood-Paley
sum (which excludes the contribution of the low-pass filter)
\begin{equation}
  \mathcal{N} = \max_{\bm{\omega}} \frac{1}{2} \sum_{j \leq J} \sum_{r \in R}
  \left( \left| \hat{\psi}_{a^j, r}(\bm{\omega}) \right|^2 + \left| \hat{\psi}_{a^j, r}(\bm{-\omega}) \right|^2
  \right).
\end{equation}


\section{Discretizing $SO(3)$}

We still need to specify which rotations we will use to build our filter bank, and, consequently, what the correct
values of $\xi$ and $\sigma$ are to satisfy \autoref{eq:lp_condition}. Thanks to the Morlet wavelet having a real
Fourier transform and our CT images being real, we only have to choose a finite number of positive rotations so that our
scaled and rotated wavelets cover (are non-zero) on as much of the frequencies
\begin{equation}\label{eq:positive_halfspace}
(\omega_x, \omega_y, \omega_z) \qquad 0 < \omega_x < \pi, -\pi < \omega_y, \omega_z < \pi
\end{equation}
where $\pi$ is the Nyquist frequency in radians (equal to $N/2$ for an $N \times N \times N$ image). In practice, we can
impose the additional constraint
\begin{equation}\label{eq:freq_ball}
    \omega_{x}^2 + \omega_{y}^2 + \omega_{z}^2 \leq \pi^2
\end{equation}
since we don't care so much about the highest possible frequencies in the signal (they are very small anyway).

A scaled and rotated wavelet has Fourier support around $\bm{\omega_c} = a^{-j}r\bm{\xi}$ with bandwidth $b_{a^j} =
a^{-j}b$. Hence, because $\bm{\xi} \propto (1, 0, 0)$ and we essentially want to cover the half-ball described by
\autoref{eq:positive_halfspace} and \autoref{eq:freq_ball}, we should choose our rotations $r$ such that they map the
unit vector $(1, 0, 0)$ onto $n$ points on the hemisphere with north pole $(1, 0, 0)$ such that the pairwise distance in
$\mathbb{R}^3$ is maximal (i.e. they are \enquote{evenly spread out} across the hemisphere).

Placing our points on the corners of a platonic solid is ideal, but only works for at most 10 points (the regular
dodecahedron). For a solution for general $n$, I have used the Fibonacci lattice, which distributes $n$ points to
maximize their distance \emph{along the sphere} in an approximately optimal way, but this should be a good approximation
to the problem at hand when to number of points is large.



\appendix

\chapter{Fast implementation}

\section{Downsampling in the Fourier domain}
I downsample in the Fourier domain by a factor $2^j$ by simply setting to zero all frequencies
outside the range $[-\frac{\pi}{2^j}, \frac{\pi}{2^j}]$ in each dimension. This ideal low-pass filter corresponds to a
convolution with a normalized sinc-filter in the spatial domain.

For a discrete-time signal $x[n]$, the Fourier transform is
\begin{equation}
  X[\omega] = \sum_{n = -\infty}^{\infty} x[n] e^{-i\omega n}.
\end{equation}
Downsampling the spatial signal $x$ by a factor $D$, i.e.
\begin{equation}
  x_D[n] = x[Dn]
\end{equation}
corresponds in the Fourier domain to
\begin{equation}
  X_D[\omega] = \frac{1}{D}\sum_{k=0}^{D-1}X(\frac{\omega - 2 \pi k}{D}).
\end{equation}





\printbibliography

\end{document}
